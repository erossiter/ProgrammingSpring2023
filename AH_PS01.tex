% Options for packages loaded elsewhere
\PassOptionsToPackage{unicode}{hyperref}
\PassOptionsToPackage{hyphens}{url}
\PassOptionsToPackage{dvipsnames,svgnames,x11names}{xcolor}
%
\documentclass[
]{article}
\usepackage{amsmath,amssymb}
\usepackage{lmodern}
\usepackage{iftex}
\ifPDFTeX
  \usepackage[T1]{fontenc}
  \usepackage[utf8]{inputenc}
  \usepackage{textcomp} % provide euro and other symbols
\else % if luatex or xetex
  \usepackage{unicode-math}
  \defaultfontfeatures{Scale=MatchLowercase}
  \defaultfontfeatures[\rmfamily]{Ligatures=TeX,Scale=1}
\fi
% Use upquote if available, for straight quotes in verbatim environments
\IfFileExists{upquote.sty}{\usepackage{upquote}}{}
\IfFileExists{microtype.sty}{% use microtype if available
  \usepackage[]{microtype}
  \UseMicrotypeSet[protrusion]{basicmath} % disable protrusion for tt fonts
}{}
\makeatletter
\@ifundefined{KOMAClassName}{% if non-KOMA class
  \IfFileExists{parskip.sty}{%
    \usepackage{parskip}
  }{% else
    \setlength{\parindent}{0pt}
    \setlength{\parskip}{6pt plus 2pt minus 1pt}}
}{% if KOMA class
  \KOMAoptions{parskip=half}}
\makeatother
\usepackage{xcolor}
\usepackage[margin=1in]{geometry}
\usepackage{color}
\usepackage{fancyvrb}
\newcommand{\VerbBar}{|}
\newcommand{\VERB}{\Verb[commandchars=\\\{\}]}
\DefineVerbatimEnvironment{Highlighting}{Verbatim}{commandchars=\\\{\}}
% Add ',fontsize=\small' for more characters per line
\usepackage{framed}
\definecolor{shadecolor}{RGB}{248,248,248}
\newenvironment{Shaded}{\begin{snugshade}}{\end{snugshade}}
\newcommand{\AlertTok}[1]{\textcolor[rgb]{0.94,0.16,0.16}{#1}}
\newcommand{\AnnotationTok}[1]{\textcolor[rgb]{0.56,0.35,0.01}{\textbf{\textit{#1}}}}
\newcommand{\AttributeTok}[1]{\textcolor[rgb]{0.77,0.63,0.00}{#1}}
\newcommand{\BaseNTok}[1]{\textcolor[rgb]{0.00,0.00,0.81}{#1}}
\newcommand{\BuiltInTok}[1]{#1}
\newcommand{\CharTok}[1]{\textcolor[rgb]{0.31,0.60,0.02}{#1}}
\newcommand{\CommentTok}[1]{\textcolor[rgb]{0.56,0.35,0.01}{\textit{#1}}}
\newcommand{\CommentVarTok}[1]{\textcolor[rgb]{0.56,0.35,0.01}{\textbf{\textit{#1}}}}
\newcommand{\ConstantTok}[1]{\textcolor[rgb]{0.00,0.00,0.00}{#1}}
\newcommand{\ControlFlowTok}[1]{\textcolor[rgb]{0.13,0.29,0.53}{\textbf{#1}}}
\newcommand{\DataTypeTok}[1]{\textcolor[rgb]{0.13,0.29,0.53}{#1}}
\newcommand{\DecValTok}[1]{\textcolor[rgb]{0.00,0.00,0.81}{#1}}
\newcommand{\DocumentationTok}[1]{\textcolor[rgb]{0.56,0.35,0.01}{\textbf{\textit{#1}}}}
\newcommand{\ErrorTok}[1]{\textcolor[rgb]{0.64,0.00,0.00}{\textbf{#1}}}
\newcommand{\ExtensionTok}[1]{#1}
\newcommand{\FloatTok}[1]{\textcolor[rgb]{0.00,0.00,0.81}{#1}}
\newcommand{\FunctionTok}[1]{\textcolor[rgb]{0.00,0.00,0.00}{#1}}
\newcommand{\ImportTok}[1]{#1}
\newcommand{\InformationTok}[1]{\textcolor[rgb]{0.56,0.35,0.01}{\textbf{\textit{#1}}}}
\newcommand{\KeywordTok}[1]{\textcolor[rgb]{0.13,0.29,0.53}{\textbf{#1}}}
\newcommand{\NormalTok}[1]{#1}
\newcommand{\OperatorTok}[1]{\textcolor[rgb]{0.81,0.36,0.00}{\textbf{#1}}}
\newcommand{\OtherTok}[1]{\textcolor[rgb]{0.56,0.35,0.01}{#1}}
\newcommand{\PreprocessorTok}[1]{\textcolor[rgb]{0.56,0.35,0.01}{\textit{#1}}}
\newcommand{\RegionMarkerTok}[1]{#1}
\newcommand{\SpecialCharTok}[1]{\textcolor[rgb]{0.00,0.00,0.00}{#1}}
\newcommand{\SpecialStringTok}[1]{\textcolor[rgb]{0.31,0.60,0.02}{#1}}
\newcommand{\StringTok}[1]{\textcolor[rgb]{0.31,0.60,0.02}{#1}}
\newcommand{\VariableTok}[1]{\textcolor[rgb]{0.00,0.00,0.00}{#1}}
\newcommand{\VerbatimStringTok}[1]{\textcolor[rgb]{0.31,0.60,0.02}{#1}}
\newcommand{\WarningTok}[1]{\textcolor[rgb]{0.56,0.35,0.01}{\textbf{\textit{#1}}}}
\usepackage{graphicx}
\makeatletter
\def\maxwidth{\ifdim\Gin@nat@width>\linewidth\linewidth\else\Gin@nat@width\fi}
\def\maxheight{\ifdim\Gin@nat@height>\textheight\textheight\else\Gin@nat@height\fi}
\makeatother
% Scale images if necessary, so that they will not overflow the page
% margins by default, and it is still possible to overwrite the defaults
% using explicit options in \includegraphics[width, height, ...]{}
\setkeys{Gin}{width=\maxwidth,height=\maxheight,keepaspectratio}
% Set default figure placement to htbp
\makeatletter
\def\fps@figure{htbp}
\makeatother
\setlength{\emergencystretch}{3em} % prevent overfull lines
\providecommand{\tightlist}{%
  \setlength{\itemsep}{0pt}\setlength{\parskip}{0pt}}
\setcounter{secnumdepth}{-\maxdimen} % remove section numbering
\ifLuaTeX
  \usepackage{selnolig}  % disable illegal ligatures
\fi
\IfFileExists{bookmark.sty}{\usepackage{bookmark}}{\usepackage{hyperref}}
\IfFileExists{xurl.sty}{\usepackage{xurl}}{} % add URL line breaks if available
\urlstyle{same} % disable monospaced font for URLs
\hypersetup{
  pdftitle={Problem Set 1},
  colorlinks=true,
  linkcolor={Maroon},
  filecolor={Maroon},
  citecolor={Blue},
  urlcolor={blue},
  pdfcreator={LaTeX via pandoc}}

\title{Problem Set 1}
\author{}
\date{\vspace{-2.5em}Due January 31, 2023}

\begin{document}
\maketitle

\hypertarget{instructions}{%
\subsection{Instructions}\label{instructions}}

\begin{itemize}
\tightlist
\item
  Read all of these instructions closely.
\item
  This problem set is due Tuesday, January 31, 2023 at 4pm. We will talk
  in class on Monday, January 30 about how to submit via Github.
\item
  Submit files via Github:

  \begin{enumerate}
  \def\labelenumi{\arabic{enumi}.}
  \tightlist
  \item
    the .Rmd (R Markdown) file
  \item
    the knitted .pdf file
  \item
    anything else the particular problem set might require
  \end{enumerate}
\item
  Use a copy of this file, perhaps with your name or initials appended
  to the file name, to write your answers to the questions. You'll see
  there is a designated space where your answers should begin.
\item
  Knitting the .Rmd file to a .pdf file \emph{as you work} will ensure
  your code runs without errors and is working how you expect. Knit
  early and often. You've already read the instruction that a knitted
  .pdf is required when you submit. (Question 1 practices this.)
\item
  Per the syllabus, I will not accept any late work. Keep in mind the
  two lowest problem set scores are dropped. Turn in what you have.
\end{itemize}

\hypertarget{question-1getting-started-with-rmarkdown}{%
\section{Question 1--Getting Started with
RMarkdown}\label{question-1getting-started-with-rmarkdown}}

This question will help orient you to RMarkdown if you haven't used it
before.

\hypertarget{a}{%
\subsection{1a}\label{a}}

In your own words, what is the difference between R, RStudio, and
RMarkdown?

Answer: R is a free, open-source programming language used to manage
data sets, create simulations, and create visualizations and models
based on that data.

RStudio is an environment used to make R more user-friendly. The
platform makes managing collections of code easier to edit and write,
while the actual R processes are still done in R.

RMarkdown is a type of R file that allows you to knit the document into
other file types, while also allowing the inclusion of normal
(non-coding) text.

\hypertarget{b}{%
\subsection{1b}\label{b}}

When we're working in an RMarkdown file (like this one), we can easily
write text around our R code. To distinguish was is R code, we put in
what is called ``code chunks.'' To see how it works, put \texttt{2+2} in
the grey area, or ``code chunk,'' below. Then, run the code. Make sure
you see the answer \emph{both} in the RStudio console and right below
the code chunk.

Answer:

\begin{Shaded}
\begin{Highlighting}[]
\CommentTok{\#put your code here}
\DecValTok{2}\SpecialCharTok{+}\DecValTok{2}
\end{Highlighting}
\end{Shaded}

\begin{verbatim}
## [1] 4
\end{verbatim}

\hypertarget{c}{%
\subsection{1c}\label{c}}

We can edit and run code as we're working in the .Rmd file in RStudio
(that is what we did in 1b). We can further ``knit'' the file into a
.pdf. Knitting will compile the text and run the code you've written,
starting at the top of the file from scratch and working its way down.
(This means it doesn't have access to objects in the global environment.
Everything you need must be written in the .Rmd).

Before proceeding any further in the assignment, try knitting. Click the
``Knit'' button in RStudio and a .pdf of the same name should be created
in the location of where this .Rmd is saved. If you can't find the
``Knit'' button, it is displayed as \#3 on
\href{https://rmarkdown.rstudio.com/lesson-15.HTML}{this cheatsheet}.

Find the .pdf file and check that the R code in 1b was executed and is
correct.

Note:

\begin{enumerate}
\def\labelenumi{\arabic{enumi}.}
\tightlist
\item
  You may need to install an additional library or two for the .pdf to
  knit. I've included that code below. Ping me in Slack as needed to
  debug any issues you might encounter on your individual computer.
\item
  If you want to see someone walk through these steps, I recommend
  \href{https://www.youtube.com/watch?v=oCK69u_yWQY}{this short video}.
\end{enumerate}

\begin{Shaded}
\begin{Highlighting}[]
\FunctionTok{library}\NormalTok{(tinytex)}
\end{Highlighting}
\end{Shaded}

\hypertarget{question-2agility-with-vectors-and-functions}{%
\section{Question 2--Agility with Vectors and
Functions}\label{question-2agility-with-vectors-and-functions}}

We learned several functions in class, like \texttt{rep} and
\texttt{seq}. This question asks you to use a few more, like
\texttt{paste0}, to practice working with vectors and functions.
Developing agility with these R basics will help you with broader skills
like cleaning data.

Hint: Look at the helpfiles for these functions if they are new to you
by typing, for example, \texttt{?paste0} in the console and viewing the
file that populates in the ``Help'' tab in the bottom right of RStudio.

\hypertarget{a-1}{%
\subsection{2a}\label{a-1}}

Create a vector called \texttt{x} of the elements 1 to 100 (i.e.,
\([1,100]\)). Then, make every odd-indexed element of the vector \(x\)
negatively signed.

\begin{Shaded}
\begin{Highlighting}[]
\NormalTok{x }\OtherTok{\textless{}{-}}\NormalTok{ (}\DecValTok{1}\SpecialCharTok{:}\DecValTok{100}\NormalTok{)}
\FunctionTok{ifelse}\NormalTok{(x }\SpecialCharTok{\%\%} \DecValTok{2} \SpecialCharTok{!=} \DecValTok{0}\NormalTok{, x}\SpecialCharTok{*{-}}\DecValTok{1}\NormalTok{, x)}
\end{Highlighting}
\end{Shaded}

\begin{verbatim}
##   [1]  -1   2  -3   4  -5   6  -7   8  -9  10 -11  12 -13  14 -15  16 -17  18
##  [19] -19  20 -21  22 -23  24 -25  26 -27  28 -29  30 -31  32 -33  34 -35  36
##  [37] -37  38 -39  40 -41  42 -43  44 -45  46 -47  48 -49  50 -51  52 -53  54
##  [55] -55  56 -57  58 -59  60 -61  62 -63  64 -65  66 -67  68 -69  70 -71  72
##  [73] -73  74 -75  76 -77  78 -79  80 -81  82 -83  84 -85  86 -87  88 -89  90
##  [91] -91  92 -93  94 -95  96 -97  98 -99 100
\end{verbatim}

\hypertarget{b-1}{%
\subsection{2b}\label{b-1}}

Use the \texttt{seq()} and \texttt{paste0()} functions to create the
vector called \texttt{varnames} with the following structure:
\texttt{c("Var10",\ "Var20",\ "Var30",\ "Var40",\ "Var50",\ "Var60")}

\begin{Shaded}
\begin{Highlighting}[]
\NormalTok{varnames }\OtherTok{\textless{}{-}} \FunctionTok{paste0}\NormalTok{(}\StringTok{"Var"}\NormalTok{, }\FunctionTok{seq}\NormalTok{(}\DecValTok{10}\NormalTok{,}\DecValTok{60}\NormalTok{,}\DecValTok{10}\NormalTok{))}

\NormalTok{varnames}
\end{Highlighting}
\end{Shaded}

\begin{verbatim}
## [1] "Var10" "Var20" "Var30" "Var40" "Var50" "Var60"
\end{verbatim}

\hypertarget{c-1}{%
\subsection{2c}\label{c-1}}

Using the vector you created in 2b, write code that removes the leading
``Var'' from each element of the vector. Call the new vector
\texttt{varnames\_str}, and the resulting vector should be the following
strings: \texttt{c("10",\ "20",\ "30",\ "40",\ "50",\ "60")}

Hint: you could use the \texttt{substr} function.

\begin{Shaded}
\begin{Highlighting}[]
\NormalTok{varnames\_str }\OtherTok{\textless{}{-}} \FunctionTok{substr}\NormalTok{(varnames, }\DecValTok{4}\NormalTok{,}\DecValTok{5}\NormalTok{)}
\CommentTok{\#I had never used substr before, http://rfunction.com/archives/1692 explained it to me before I typed the code}
\NormalTok{varnames\_str}
\end{Highlighting}
\end{Shaded}

\begin{verbatim}
## [1] "10" "20" "30" "40" "50" "60"
\end{verbatim}

\hypertarget{d}{%
\subsection{2d}\label{d}}

Use the vector \texttt{varnames\_str} created in 2c and change the
string elements into numeric values, naming this vector
\texttt{varnames\_num}. The new vector should look like this:
\texttt{c(10,\ 20,\ 30,\ 40,\ 50,\ 60)}

\begin{Shaded}
\begin{Highlighting}[]
\NormalTok{varnames\_num }\OtherTok{\textless{}{-}} \FunctionTok{as.numeric}\NormalTok{(varnames\_str)}
\NormalTok{varnames\_num}
\end{Highlighting}
\end{Shaded}

\begin{verbatim}
## [1] 10 20 30 40 50 60
\end{verbatim}

\hypertarget{e}{%
\subsection{2e}\label{e}}

Create a subset the vector \texttt{varnames\_num} from 2d that only
includes the numbers divisible by 4. Name this vector
\texttt{varnames\_d4}.

Hint: you might use the modulo function, \texttt{\%\%}.

\begin{Shaded}
\begin{Highlighting}[]
\NormalTok{varnames\_d4 }\OtherTok{\textless{}{-}}\NormalTok{ varnames\_num[varnames\_num}\SpecialCharTok{\%\%}\DecValTok{4}\SpecialCharTok{==}\DecValTok{0}\NormalTok{]}

\NormalTok{varnames\_d4}
\end{Highlighting}
\end{Shaded}

\begin{verbatim}
## [1] 20 40 60
\end{verbatim}

\hypertarget{f}{%
\subsection{2f}\label{f}}

Uncomment and run the following code and describe exactly what is
happening (what R functionality and calculations) to get this result.

Hint: it might help to write the math out for each element.

\begin{Shaded}
\begin{Highlighting}[]
\NormalTok{varnames\_num }\SpecialCharTok{{-}}\NormalTok{ varnames\_d4}
\end{Highlighting}
\end{Shaded}

\begin{verbatim}
## [1] -10 -20 -30  20  10   0
\end{verbatim}

\begin{Shaded}
\begin{Highlighting}[]
\CommentTok{\#Varnames\_num is c(10,20,30,40,50,60)}
\CommentTok{\#varnames\_d4 is c(20,40,60)}

\CommentTok{\#This code is utilizing the fact that math is vectorized in R. The matching placements of a vector (position 1 and position 1, position 2 and position 2, etc) are being subtracted from each other. Here, the number of elements do not match. The larger vector has 6, and the smaller vector has 3. As we talked about in class last week, R automatically begins using the smaller vector again. }

\CommentTok{\#Here, varnames\_num[1] {-} varnames\_d4[1] = {-}10, followed by the second and third positions ({-}20, {-}30, respectively). Then, the shorter vector\textquotesingle{}s first position is used again. varnames\_num[4] {-} varnames\_d4[1] = 20. This pattern is followed until the subtraction has happened for every element of varnames\_num. }
\end{Highlighting}
\end{Shaded}

\hypertarget{question-3math-in-r}{%
\section{Question 3--Math in R}\label{question-3math-in-r}}

Like the inclass activity, this question asks you to practice
translating math equations into R. While you might not plan to do a lot
of math in R in the future, this question helps make R rules and syntax
second-nature e.g., closing all open parentheses, knowing how to work
with vectorized operations, etc.

Hint: look at the compiled .pdf file I provided to see the math in an
easily-readable format and ignore the code between the pink dollar
signs.

\hypertarget{a-2}{%
\subsection{3a}\label{a-2}}

\(\frac{3}{7} \times2^{\frac{1}{5}} \times \sum_{i=1}^{50} i\)

\begin{Shaded}
\begin{Highlighting}[]
\CommentTok{\#code here}
\end{Highlighting}
\end{Shaded}

\hypertarget{b-2}{%
\subsection{3b}\label{b-2}}

\(\sum_{i=1}^{50} (i+10)\)

\begin{Shaded}
\begin{Highlighting}[]
\CommentTok{\#code here}
\end{Highlighting}
\end{Shaded}

\hypertarget{c-2}{%
\subsection{3c}\label{c-2}}

\(\forall i \in [1,10],\; i^{\sqrt{i+2}}\)

\begin{Shaded}
\begin{Highlighting}[]
\CommentTok{\#code here}
\end{Highlighting}
\end{Shaded}

\hypertarget{question-4writing-functions}{%
\subsection{Question 4--Writing
Functions}\label{question-4writing-functions}}

\hypertarget{a-3}{%
\subsection{1a}\label{a-3}}

Write a function named \texttt{sample\_var} that takes one input, a
numeric vector \texttt{x}, and calculates the standard sample variance
formula. Have the function return the answer.

\[\frac{1}{n-1} \sum_{i=1}^n (x_i - \bar{x})^2\]

\begin{Shaded}
\begin{Highlighting}[]
\CommentTok{\#code here}
\end{Highlighting}
\end{Shaded}

\hypertarget{b-3}{%
\subsection{1b}\label{b-3}}

Use the function you wrote in 1a to calculate the sample variance of the
following vector. You can use the built-in function in R to double check
your function is correct

\begin{Shaded}
\begin{Highlighting}[]
\NormalTok{vec }\OtherTok{\textless{}{-}} \FunctionTok{c}\NormalTok{(}\DecValTok{2}\NormalTok{, }\DecValTok{5}\NormalTok{, }\DecValTok{10}\NormalTok{, }\DecValTok{12}\NormalTok{, }\DecValTok{3}\NormalTok{, }\DecValTok{3}\NormalTok{, }\DecValTok{4}\NormalTok{, }\DecValTok{1}\NormalTok{, }\DecValTok{20}\NormalTok{)}
\FunctionTok{var}\NormalTok{(vec)}
\end{Highlighting}
\end{Shaded}

\begin{verbatim}
## [1] 38.5
\end{verbatim}

\begin{Shaded}
\begin{Highlighting}[]
\CommentTok{\#sample\_var(vec) \# this should return same answer as var()}
\end{Highlighting}
\end{Shaded}

\hypertarget{c-3}{%
\subsection{1c}\label{c-3}}

Now, try running your function on the following vector. Why won't it
work?

\begin{Shaded}
\begin{Highlighting}[]
\NormalTok{vec\_str }\OtherTok{\textless{}{-}} \FunctionTok{c}\NormalTok{(}\StringTok{"2"}\NormalTok{, }\StringTok{"5"}\NormalTok{, }\StringTok{"10"}\NormalTok{, }\StringTok{"12"}\NormalTok{, }\StringTok{"3"}\NormalTok{, }\StringTok{"3"}\NormalTok{, }\StringTok{"4"}\NormalTok{, }\StringTok{"1"}\NormalTok{, }\StringTok{"20"}\NormalTok{)}
\end{Highlighting}
\end{Shaded}


\end{document}
